"Vậy thì", cái tạo vật nho nhỏ kia nói, trong lúc những đứa còn lại tủa ra từ khắp nơi đến sau lưng nó, "ông đi hỏi Cha Horrigan cho chúng tôi biết là linh hồn chúng tôi có được cứu rỗi vào ngày phán xét, như bao tín hữu Kitô ngoan đạo khác không nhé; và nếu ông rủ lòng thương chúng tôi, hãy đừng chần chừ mà mang lời cha trở lại ngay".

Ông Dermod trở lại túp lều, thấy khoai tây đã dọn sẵn trên bàn ăn, còn người vợ thảo của mình thì bưng cái củ to tướng nhất, một trái táo đỏ tươi rói, phì khói như con ngựa chạy kiệt lực trong đêm sương, đến cho Cha Horrigan.

"Thưa cha", ông Dermod nói, sau một thoáng chần chừ, "cho con mạn phép được hỏi cha kính yêu một câu này được không ạ?"

"Là gì đó con?" Cha Horrigan nói.

"Vậy, thì, xin cha kính yêu thứ lỗi vì nếu có quá đường đột, chẳng là, Liệu linh hồn của những người tốt tính có được cứu rỗi vào ngày phán xét không ạ?"

"Ai xui con thưa với ta như thế vậy, hả Leary?" vị linh mục nói, nghiêm khắc chằm chặp nhìn ông, khiến ông Dermod không thể nào chịu đựng nổi.

"Con sẽ không dối cha chuyện này đâu, trừ sự thật thì con sẽ không nói dối chuyện gì trên đời hết", ông Dermod nói. "Do những người tốt tính gửi gắm con câu hỏi này đó, có cả ngàn đứa đang dưới chỗ bờ sông đợi con mang câu trả lời về đó ạ".