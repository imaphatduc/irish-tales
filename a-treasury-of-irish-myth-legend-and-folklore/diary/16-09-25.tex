Trong các mái đầu hồi nhà Frank có một cái xây quay lưng lại, mà hình như là trông ra, một khu Pháo Đài hoặc Đồ Trận gì đó, gọi là Towny mà chuẩn chỉnh hơn là Pháo Đài Tonagh. Người ta nói cả khu ấy bị tiên ám, và cái đã gây nên cái bản chất hoang dại kỳ quái của nó trong mắt tôi là, ở mạn phía Nam của nó có hai ba khu đất gò màu xanh nho nhỏ, nghe kể là nơi chôn trẻ con chưa được rửa tội mà nếu đi quá nữa sẽ nhận hết hiểm nguy và xui rủi vào người. Gì thì gì, đất trời đương tiết Hạ chí; và vào một tối tầm chạng vạng, cơn bệnh đứa trẻ còn chưa dứt, thì có tiếng cưa tay phát ra từ khu Pháo Đài. Thật là lạ, rồi sau đó một lúc, vài người đang tụ tập tại nhà Frank chạy ra xem ai lại đang cưa ở cái chốn như thế, ai lại đang cưa vào cái giấc trễ thế này, vì ai ai cũng biết không một ai sống ở đất nước này lại dám đi cưa cây gai trắng mọc trên đất Pháo Đài được. Tuy vậy, trên đường ra kiểm tra, mà hãy đoán xem họ sửng sốt thế nào đi, sau một hồi đi vòng quanh lùng sục cả nơi ấy, họ không tìm ra vết cưa hay cái cưa nào hết. Thật ra, ngoài chính bản thân họ thì không còn ai, cả tự nhiên lẫn siêu nhiên, là hữu hình hết. Sau đó họ trở lại căn nhà, còn chưa kịp ngồi xuống, thì lại có tiếng ấy phát ra trong tầm mười thước tính từ chỗ họ. Cuộc kiểm tra lại được tiến hành, nhưng mức độ thành công thì cũng giống lần trước. Tuy nhiên, giờ đây, đứng trên khu Pháo Đài, họ nghe thấy tiếng cưa trong một cái hố nhỏ, cỡ khoảng một trăm năm mươi thước bên dưới chỗ họ đứng, hoàn toàn trong tầm nhìn được, tuy vậy họ lại không thấy ai hết. Lập tức, một đám nhỏ đi xuống, nếu được, để hòng cắt nghĩa cái tiếng động kỳ lạ này và cái công việc vô hình kia; nhưng khi đến chỗ ấy, họ nghe thấy tiếng cưa, giờ lại thêm tiếng búa đập, và tiếng đóng đinh từ khu Pháo Đài bên trên kia nữa, mặc cho những người đứng phía trên khu Pháo Đài lại nghe thấy âm thanh phát ra từ bên dưới. Chín người mười ý xong, họ mới nhất trí cho người đến nhà Billy Nelson gọi Frank Martin, cách đó chỉ chừng tám chín mươi thước gì đó. Sau đó không lâu thì anh tới, và không mất chút thì giờ nao núng nào anh đã giải cái bí ẩn này được ngay.

"Tụi tiên đó", anh nói. "Tôi thấy tụi nó, tụi tạo vật bận rộn".

"Nhưng chúng cưa cái gì vậy Frank?"

"Đóng hòm cho con nít", anh đáp; "chúng nhờ làm được cái xác rồi, và giờ thì đóng đinh hai mép hòm lại".

Đêm đó đứa bé ấy đã chết, và chuyện lại tiếp diễn, rằng vào buổi tối thứ hai sau đó, người thợ mộc đảm trách việc đóng hòm đã mang một cái bàn từ nhà Thomas ra chỗ Pháo Đài để dùng tạm làm một băng ghế; và, nghe kể, rằng tiếng cưa gỗ và búa đập cần thiết để hoàn thành công việc của ông ta lại giống y như cái tiếng mà người ta nghe thấy trước đó, trừ tối hôm trước – không hơn không kém. Tôi còn nhớ cái chết của đứa bé ấy, và cả việc đóng hòm cho nó, nhưng chừng như người làng không kháo nhau nghe chuyện người thợ mộc siêu nhiên suốt vài tháng sau lễ an táng.

Nét nào trên người Frank cũng trông bệnh tật cả. Khi tôi gặp anh thì hình như anh mới khoảng ba mươi tư tuổi thôi, nhưng tôi không nghĩ, từ cái vóc dáng gầy gò và thể trạng suy nhược trước mắt tôi kia, là anh đã sống qua được vài năm rồi. Người ta hứng thú, hiếu kỳ rất nhiều về nhân vật này, mà nhiều phen tôi chứng kiến anh được giới thiệu cho khách lạ với cái danh "nhìn ra được những người tốt tính".