Và rồi đám tiên lấy hết sức ríu rít chạy biến đi, trốn sau kẽ lá tiên thảo\footnote{lusmore} xanh rờn. Nơi đây, nếu các vành nón đỏ nho nhỏ có lộ ra thì cũng chỉ trông như là chuông hoa thẫm đỏ thôi; và nhiều đứa thì nấp trong bóng cây mâm xôi và đá tảng, những đứa khác nấp dưới bờ sông, trong các hốc nứt đủ các loại.

Đứa tiên loan báo quả không sai; vì đang trên đường kia, trong tầm nhìn thấy từ bờ sông, là Cha Horrigan đang cưỡi ngựa con đến, nghĩ rằng vì đã trễ quá rồi nên cha sẽ kết thúc hành trình tại túp lều\footnote{cabin} đầu tiên mình thấy. Như quyết định, cha dừng lại trước cửa nhà ông Dermod Leary, nâng then gài cửa bước vào, "cầu phước lành cho cả nhà ta nhé"\footnote{my blessing on all here (Irish greetings)}.

Tôi cũng chẳng cần phải nói rằng, Cha Horrigan luôn được chào đón hễ nơi nào cha đến, vì khắp đất nước này không ai được kính mộ và quý mến như ông. Giờ thì ông Dermod gặp phải vấn đề lớn rồi, vì ông chẳng có gì làm đậm thêm vị khoai tây để tỏ lòng thành kính trên bàn ăn tối. "Bà già", như ông Dermod gọi vợ mình, dù tuổi cô còn chưa quá đôi mươi, đang luộc nồi khoai trên lửa; ông nghĩ về tấm lưới mình giăng trên sông, nhưng vì chưa được lâu mấy nên cơ hội nhìn thấy cá mắc lưới là không có. "Có gì đâu", ông Dermod nghĩ, "xuống đó thử cũng có hề gì; và biết đâu được, vì ta muốn có cá trong bữa tối dâng vị linh mục, nên cá sẽ nằm sẵn đó đợi ta đến thôi".

Ông Dermod xuống chỗ bờ sông, thấy trong tấm lưới có một con cá hồi đẹp chưa từng có trên mặt nước "sông Lee sải rộng" sáng trong này; nhưng khi ông đương gỡ cá thì tấm lưới giật ngược lại, ông không biết là tại làm sao, còn con cá thì chạy mất tiêu, mừng vui mà bơi xuôi dòng như chưa có gì xảy ra cả.
