Ông Dermod buồn rầu nhìn vệt nước mà con cá hồi để lại, sáng như vệt bạc trăng thâu, và rồi, bàn tay phải ông tức tối khua múa, chân giậm mạnh, để dọn đường cho nỗi lòng tuôn trào, ông mới lẩm nhẩm, "Cầu cho mày sáng tối đi đâu cũng gặp mạt vận xui rủi, cái đồ cá ranh ma, láu cá! Nếu còn tí xấu hổ nào thì mày nên tự thấy vậy đi, xỏ lá đến vậy mà coi được! Và tao chắc cú là mày sẽ không ra gì đâu, vì có thứ quỷ quyệt hay gì đó giúp mày mà – không thấy nó đang kéo giật ngược lại tao mạnh như là quỷ dữ vậy hả?"

"Không phải như vậy đâu", một tên tiên trong cái đám nho nhỏ vừa phải chạy trốn vị linh mục cất tiếng, đi đến chỗ ông Dermod Leary cùng một đám bạn hữu theo sát gót; "chỉ có một tá rưỡi đứa trong đám chúng tôi kéo giật ngược lại ông thôi".

Ông Dermod say sưa nhìn cái tạo vật nho nhỏ đang nói, nó tiếp, "Hãy tuyệt nhiên đừng để bữa tối của vị linh mục làm ông bận lòng; vì nếu ông trở lại hỏi ngài ấy giúp chúng tôi một câu này thôi, thì một bữa tối tuyệt hảo chưa từng có sẽ được bày biện trên bàn ăn dành cho ngài trong nháy mắt".

"Tôi không dính dáng gì đến mấy người hết", ông Dermod cương nghị đáp; ông dừng lời một chút rồi tiếp, "Tôi vô cùng cảm tạ với đề nghị đó, thưa ngài, nhưng tôi chẳng dại gì mà bán mình cho ngài, hay cho những loại như ngài, để đổi lấy một bữa tối; và hơn thế, tôi biết Cha Horrigan luôn nghĩ cho linh hồn của tôi hơn là mong tôi mang nó đi cầm cố mãi mãi, dù cho ngài có bày biện cho cha thứ gì đi nữa – nên là chấm hết câu chuyện ở đây".

Cái tạo vật nho nhỏ kia ngoan cố không để cho thái độ của ông Dermod lấn lướt, nói tiếp, "Ông hỏi vị linh mục một câu rất lịch sự này giúp chúng tôi thôi được không?"

Ông Dermod cân nhắc một lúc, cân nhắc là đúng, nhưng ông lại nghĩ có ai đi hỏi một câu hỏi lịch sự mà lại gặp tai ương. "Tôi không thấy có lý do gì để phản đối điều ấy hết, thưa các ngài", ông Dermod nói; "nhưng tôi không dính dáng gì đến cái bữa tối của các ngài đâu – xin nhớ dùm cho".
