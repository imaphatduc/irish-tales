\section{Frank Martin và đàn tiên}

\begin{large}
  \begin{center}
    \uppercase{William Carleton}
  \end{center}
\end{large}

Martin mang nước da nhạt, ốm gầy, mỗi lần tôi nhìn đến là lại trông xanh xao, thể chất từ bẩm sinh đã yếu đuối. Tóc anh ánh màu nâu đỏ, râu ria gần như nhẵn thín, đôi bàn tay thon mảnh, trắng trẻo khác lạ, mà tôi dám chắc cũng bởi do tính chất nhẹ nhàng, dễ thở của cái công sự anh làm cũng như do cái thể trạng suy nhược của anh. Về mọi thứ khác thì anh đều phải lẽ, tỉnh táo và lý trí như bao người khác; nhưng khi nói về đàn tiên, anh lại xúc động mạnh đến mức lạ thường, không di dịch. Tất nhiên, tôi còn nhớ cái biểu cảm trong ánh nhìn hoang dại, trống rỗng của anh, thái dương anh dài, hẹp, màu da chuyển vàng vọt, hốc hác.\par

Nào, anh chàng này không sống đời bất hạnh đâu, căn bệnh anh phải chịu đựng cũng không khởi phát từ đau đớn hay kinh hãi nốt, tuy vậy mà ai cũng có thể hồ tưởng theo hướng ấy. Trái lại, anh và đàn tiên vẫn luôn giành cho nhau những tình cảm hòa nhã nhất, và khi họ chuyện trò – mà tôi sợ là phiến diện đến kinh khủng – thì anh nhất định phải rất sung sướng, vì cái tiến trình ấy luôn đầy ắp sảng khoái và tiếng cười, mà ít nhất là từ phía anh.

"Ừ thì, Frank à, anh thấy đàn tiên đó lúc nào thế?"

"Suỵt! Có tới hai tá đứa tiên đang ở cửa tiệm (xưởng dệt) ngay lúc này lận đó. Có lão ngồi trên đỉnh khung dệt, tất tật đều lắc lư khi tôi đang dệt. Trong chúng có sầu tư, nhưng lại là những đứa tinh quái tuyệt hảo nhất trần đời, thật vậy đó. Thấy không, có đứa nữa lại đang trên cái đầu đồ\footnote{một loại bột sệt keo dính quét lên sợi tơ để giữ sợi tròn đều, đồng thời ngăn không cho sợi bị sờn do ma sát với lược dệt} tôi kìa. Đi ra đi, cái tên \textit{shingawn} kia; không thì, coi như ta xui, nếu ngươi không ra, ta cho một trận bây giờ. Ê, ngưng, cái đồ ăn cắp!"

"Frank ơi, anh không sợ tụi đó hả?"

"Tôi hả! À ha, tôi sợ chúng làm cái gì? Chắc cú là chúng không làm gì được tôi hết."

"Và sao lại không vậy Frank?"

"Vì tôi được rửa tội khỏi chúng rồi."

"Ý là sao?"

"Sao hả, là vị linh mục đặt tên thánh cho tôi, cha tôi có yêu cầu ông, dâng lời nguyện xua tụi tiên đi – mà linh mục hễ được yêu cầu thì không chối từ – nên ông làm theo. Cha chả, được thế là tốt cho tôi – (ê, tha cho cái tráp mỡ đi, cái đồ tham ăn tục uống – thấy chưa, tên ăn cắp tí hon đó ăn vụng tráp mỡ của tôi kìa) – vì, như bạn thấy đó, chúng tính tôn tôi lên làm vua tiên mà."

"Được luôn hả?"

"Xạo thì quỷ bắt tôi đi. Bạn cứ đi hỏi chúng, rồi chúng nói bạn nghe."

"Chúng lớn nhỏ thế nào vậy Frank?"

"Ôi xời, tụi nhỏ xíu xiu, áo choàng xanh, mang mấy đôi giày nhỏ đẹp nhất trần đời. Có hai đứa – cả hai đều quen thân tôi từ xưa xửa – chạy dóc dọc cái trục dệt. Ông bạn già mang bộ mao giả đoạn\footnote{nguyên gốc: \textit{bob-wig}} tên Jim Jam, còn gã kia, đội nón ba góc, tên Nickey Nick. Nickey biết thổi kèn ống luôn đó. Ê, Nickey, cho tụi này một điệu coi, không thì ta phạt nặng\footnote{\textit{malavogue} (tiếng Ailen cổ)} nhà ngươi bây giờ – đến đây liền coi, cho bài 'Lough Erne Shore' đi. Suỵt, giờ thì – nghe thôi!"

Cái chàng tội nghiệp, tuy thì lúc nào cũng lấy hết sức kéo sợi cho thật nhanh, lại dốc hết lòng sao sát dõi theo tiếng nhạc, nhìn như đang hết mực thưởng thức như thể đó là thật vậy.

Nhưng ai mà biết được, phải chăng những gì ta cho là mất mát lại như một dòng suối xiết dâng hạnh phúc, hay có lẽ còn kỳ vĩ hơn tất thảy những gì làm ta thích chí nữa? Tôi quên mất tên người thi sĩ đã viết nên những dòng này rồi –

"Kỳ bí thay luật lệ ngươi ban;
Mộng tưởng còn trong hơn sơn hà,
Phong cảnh nàng Thiên Nhiên chưa hoạ
Nét mỹ kiều phác nét từ Tâm Quan".

Có nhiều lần, hồi còn nhỏ, còn chưa lên sáu hay bảy nữa, thì tôi đã lội xa tới cỡ xưởng dệt của Frank, con tim nửa tò mò nửa run sợ, để mà nghe anh trò chuyện với những con người tốt tính. Suốt từ sáng đến khuya lưỡi anh cứ chạy liến thoắng gần giống con thoi dệt vậy; người ta biết là vào khuya, cứ mỗi khi chợt tỉnh giấc thì trước nhất anh đều đưa tay ra rồi, mà trông như, là đẩy chúng xuống khỏi giường vậy.

"Đi ra đi, cái đồ đầu trộm đuôi cướp, là ngươi đó – đi ra ngay đi, để ta yên. Ê Nickey, đây là lúc để ngươi thổi nhạc đó hả, trong khi ta muốn đi ngủ? Đi ra, ngay – thề nếu mài vậy, mai mài sẽ thấy tau làm gì mài. Tất nhiên tau sẽ quay thêm đồ mặc; và nếu mài cư xử đàng hoàng, có thể tau sẽ cạo đáy nồi cho mài vài miếng. Thôi mà! Úi! Tội nghiệp, cái giống lòi tủ tế này. Tất nhiên chúng đi hết rồi, còn sít sót có thằng Nón Đỏ tội nghiệp này thôi, nó không muốn bỏ ta đi đó mà". Và rồi cái anh chàng cuồng loạn vô hại ấy lại chìm vào cái mà chúng tôi tin là giấc mộng mị.

Vào quãng ấy nghe nói có chuyện gì rất đáng chú ý khiến cho khắp xóm ai cũng coi trọng Frank. Anh chàng tên Frank Thomas, chính cái người tôi gặp ở căn nhà mà ông Mickey M'Rorey đã nhảy điệu chào sàn, tỉ mỉ như bản phác trước đó vậy; anh này, xin thưa, có đứa con bệnh ở nhà, bệnh gì giờ tôi không nhớ được, cũng không quan trọng gì .

