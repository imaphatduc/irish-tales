Tiên trong tiếng Ailen là \textit{sheehogue} [\textit{sidheóg}], tiểu hóa từ chữ "shee" trong \textit{banshee}. Tiên là \textit{denee shee} [\textit{daoine sidhe}] (người tiên).

Chúng là ai? "Là những thiên thần không tốt đẹp đến mức phải cứu vớt, cũng không xấu xa đến nỗi phải đọa đày," nghe tá điền kể là vậy. "Là thần thổ địa", Sách Armagh nói thế. "Là các vị thánh ngoại giáo người Ailen," mà theo lời giới cổ học Ailen "là thần tộc \textit{Tuatha De Danān}, khi không còn được tôn thờ, cúng bái nữa thì tiêu hoại dần trong tâm thức dân gian, đến nay tầm vóc chỉ còn vài gang tay".

Và để chứng minh thì chúng sẽ nói cho bạn biết rằng các trưởng hội đàn tiên có tên trùng với các vị anh hùng \textit{Danān} hồi xưa xửa, về nơi chúng thường đến tụ hội, các khu chôn cất \textit{Danān}, rồi thì tộc \textit{Tuath De Danān} xưa kia từng được gọi là \textit{slooa-shee} [\textit{sheagh sidhe}] (tiên hội trì) hoặc \textit{Marcra shee} (đoàn tiên kỵ hành).

Mặt khác, có rất nhiều bằng cớ cho thấy chúng là các thiên thần sa ngã. Chứng kiến bản chất của giống loài ấy, cái tâm tính thất thường của chúng, cái lối đối đãi hiền với kẻ hiền và ác với kẻ ác, sức quyến rũ có thừa nhưng thiếu lương tâm, nhất quán. Là các sinh vật dễ phật lòng đến mức bạn tuyệt đối không được nhắc đến chúng quá nhiều, với cả ngoài cái danh "quý nhân" thì đừng bao giờ gọi chúng là cái gì khác, hoặc không thì là \textit{daoine maithe} mà trong tiếng Việt mang nghĩa là người tốt tính, nhưng lại rất dễ chiều lòng, chúng sẽ làm mọi cách giúp xua đuổi bất hạnh xa khỏi bạn nếu bạn đặt một chút sữa qua đêm trên bệ cửa sổ. Tóm lại, tín ngưỡng dân gian kể cho chúng ta nghe rất nhiều về chúng rồi, kể như chúng đã sa ngã thế nào, tuy không lầm lạc, bởi cái ác tính của chúng lại tuyệt vô ác ý.

Chúng có phải "thần thổ địa" không? Có thể lắm! Rất nhiều nhà thơ cũng như tất cả những cây bút thần bí và huyền học, mọi thời đại và xuyên biên giới đã khẳng định rằng đằng sau cái hữu hình là hàng hàng lớp lớp các sinh thể có ý thức, chẳng thuộc thiên đàng mà là địa tầng, chẳng giữ hình dạng cố định mà thay đổi tùy hứng hoặc tùy vào cái tâm trí đang nhìn đến chúng. Bạn không thể nào nhấc tay lên mà không ảnh hưởng đến bầy lũ cũng như bị bầy lũ ảnh hưởng cho được. Thế giới hữu hình chỉ mới độc là lớp da của chúng. Trong giấc mơ, chúng nhan nhản xung quanh, ta bước đi, chơi đùa với chúng, kháng cự lại chúng. Có lẽ, chúng là những linh hồn con người trong lò luyện – cái loài sinh thể tùy hứng này.

Đừng có nghĩ rằng tiên thì lúc nào cũng nhỏ bé. Chúng thì cái gì cũng thất thường, ngay cả đến kích thước cũng vậy. Chúng tuồng như cứ thay hình đổi dạng sao cho dễ chịu nhất. Những công việc chính chúng làm phải kể đến như ăn nhậu tiệc tùng, đánh lộn đánh loạn rồi lại làm tình, rồi lại chơi cái thứ âm nhạc trác tuyệt nhất. Trong số chúng chỉ có một người cần mẫn, là \textit{lepra-caun} – người đóng giày. Có lẽ chúng còn mang cả giày ra để nhảy múa nữa. Gần ngôi làng Ballisodare có một cô gái nhỏ đã từng bảy năm sống chung với chúng. Khi về nhà thì cô không còn ngón chân nào nữa – cô đã nhảy múa hết cả ngón chân đi rồi.

Mỗi năm chúng đón ba ngày lễ lớn – đêm Giao Thừa tháng Năm, Đêm Giao Thừa Hạ chí và Đêm Giao Thừa tháng Mười Một. Vào cái đêm Giao Thừa tháng Năm, cứ mỗi bảy năm chúng lại đánh lộn đánh loạn cả trăm hiệp, hầu hết là ở "Lũy-khiết-Bạch"\footnote{Plain-a-Bawn} (nơi nảo nơi nao) cốt giành giật mùa màng, giành lấy vụ bông ngũ cốc về tay mình. Có một ông già từng có lần thấy chúng đánh lộn mà kể tôi nghe; rằng chúng còn phá nát cả lớp mái rơm trong lúc hỗn chiến nữa. Ai mà có đi ngang đó sẽ chỉ thấy một cơn gió lốc quét qua đang cuốn hết tất thảy lên trời. Khi cơn gió ấy cuồn cuộn rơm lá hết lên, thì các tiên và cả tá điền đều cởi nón ra rồi nói, "Cầu Chúa phù hộ chúng".

Vào cái đêm Giao Thừa Hạ chí, khi lửa trại bừng bừng trên khắp các dãy đồi để tưởng nhớ Thánh John, thì các tiên lại chìm trong cái nỗi vui sướng nhất, đôi lúc còn cướp đi nhiều người phàm xinh đẹp về làm cô dâu.

Vào cái đêm Giao Thừa tháng Mười Một thì chúng lại chìm trong cái tâm trạng não nề nhất, vì theo phép đo lường của người Gaelic thì đây là đêm đông đầu tiên. Đêm đó chúng nhảy múa cùng hồn ma, \textit{pooka} thì đi chưa về, đám phù thủy còn đang luyện phép, các cô gái thì bày biện thức ăn, nhân danh ác quỷ mà linh ảnh\footnote{fetch} người tình tương lai của họ có thể sẽ qua bệ cửa sổ mà vào chén số thức ăn ấy. Sau đêm Giao Thừa tháng Mười Một thì giàn mâm xôi đen trông không còn gì đặc sắc nữa, vì bị \textit{pooka} phá hết rồi.

Khi tức giận thì chúng sẽ phóng phi tiêu tới cho cứng đơ cả con người và đàn gia súc.

Khi vui mừng thì chúng hát. Có nhiều cô gái tội nghiệp từng nghe chúng hát, héo mòn dần rồi chết vì quá mê đắm giọng hát ấy. Có rất nhiều làn điệu Ailen cổ xưa bắt nguồn từ thứ âm nhạc của riêng chúng truyền qua tai người nghe lén. Không một tá điền còn sáng suốt nào mà lại đi ngân nga bài "Cô Gái Xinh Vắt Sữa Bò" gần các tiên đồ trận\footnote{rath} hết, vì tâm tính chúng hay ghen ghét, không ưa những cái môi phàm tục vụng về hát lên bài ca của mình. Nhà phổ sử\footnote{bard} người Ailen cuối cùng là Carolan đã từng đánh giấc trên tiên đồ trận, mà mãi sau này điệu nhạc tiên vẫn còn chạy trong đầu ông. Ông thật vĩ đại biết dường nào.

Chúng có chết không? Blake từng thấy qua lễ tang của tiên rồi; còn ở Ailen thì chúng tôi cho là chúng bất tử.
